\documentclass[a4paper,oneside,reqno]{amsart}

\input{../../../../cambridge-macros.tex}


%    Set assignment information here
\newcommand{\authorname}{Feynman Liang}
\newcommand{\coursename}{MLSALT 8: Statistical Machine Translation}
\newcommand{\assignmentname}{Practical 2: WFSTs for Language Processing}

\begin{document}

\title{\coursename\\\assignmentname}

\author{\authorname}
\date{\today}

\maketitle

\section{Preliminary Question}

\begin{figure}[ht!]
  \begin{center}
    \includegraphics[trim={0 3.0in 0 3.0in},clip,scale=0.5]{output/input.pdf}
  \end{center}
  \caption{Input acceptor}
\end{figure}

\begin{figure}[ht!]
  \begin{center}
    \includegraphics[trim={0 3.0in 0 3.0in},clip,scale=0.5]{output/model.pdf}
  \end{center}
  \caption{Model transducer}
  \label{fig:prelim-model}
\end{figure}

\begin{figure}[ht!]
  \begin{center}
    \includegraphics[trim={0 3.0in 0 3.0in},clip,scale=0.5]{output/result.pdf}
  \end{center}
  \caption{Result acceptor}
\end{figure}

\begin{enumerate}[label=(\roman*)]
  \item 4 paths, encoding 4 distinct strings.  3 strings accepted by model.
    \texttt{a b c a} is not.

    \begin{lstlisting}[language=bash]
printstrings.O2 --label-map=isyms.txt --input=input.fst -n 10 -w 2> /dev/null
a b c   0
a b a a         0.3
a b c a         0.4
a b     1
    \end{lstlisting}

  \item 4 paths, encoding 3 distinct strings.

    \begin{lstlisting}[language=bash]
printstrings.O2 --label-map=osyms.txt --input=result.fst -n 10 -w 2> /dev/null
x y x   0.7
x z x x         0.8
x z     1.3
x y x   1.6
    \end{lstlisting}

  \item 3 paths, encoding 3 distinct strings. This is because determinization leaves
    one distinct path per distinct string.
    \begin{lstlisting}[language=bash]
printstrings.O2 --label-map=osyms.txt --input=result.fst -n 10 -w 2> /dev/null
x y x   0.700195
x z x x         0.799805
x z     1.2998
    \end{lstlisting}
    \begin{figure}[ht!]
      \begin{center}
        \includegraphics[trim={0 3.0in 0 3.0in},clip,scale=0.5]{output/result2.pdf}
      \end{center}
      \caption{$\epsilon$-removed, determinized, and minimized result acceptor}
    \end{figure}

  \item In \autoref{fig:prelim-model}, the \texttt{a:x/0.1} arc forms a
    loop on accepting state $3$, so an infinite number of strings are accepted.
    For example, any input matching the regular expression ``cba$^*$'' is
    accepted by the model.
\end{enumerate}

\section{Practical Exercise}

\begin{enumerate}[label=\arabic*.]
  \item
    \begin{enumerate}[label=(\alph*)]
      \item
        We create an acceptor with a start state $0$, end state $1$, and
        arcs $0 \to 1$ accepting any symbol in $L$ (including <space>)
        except for <eps>.
        \begin{lstlisting}[language=bash]
cat $DIR/table1.txt \
  | cut -d ' ' -f 1 \
  | sed '/<eps>/d' \
  | sed -r 's/.*/0 1 & &/' \
  > 1a.txt
echo 1 >> 1a.txt
        \end{lstlisting}

      \item~\\
        \begin{lstlisting}[language=bash]
cat > 1b.txt <<EOF
0 1 <space> <space>
1
EOF
        \end{lstlisting}
        \begin{figure}[ht!]
          \begin{center}
            \includegraphics[trim={0 3.7in 0 3.7in},clip,scale=0.5]{output/1b.pdf}
          \end{center}
        \end{figure}

      \item \texttt{upper\_fst} accepts a single uppercase character:
        \begin{lstlisting}[language=bash]
upper_fst=$((cat $SYMBOLS \
  | cut -d ' ' -f 1 \
  | sed '/^[A-Z]/!d' \
  | sed -r 's/.*/0 1 & &/' \
  ; echo '1') \
  | fstcompile --isymbols=$SYMBOLS --osymbols=$SYMBOLS -)
        \end{lstlisting}

        \texttt{lower\_star\_fst} accepts zero or more lowercase characters:
        \begin{lstlisting}[language=bash]
lower_star_fst=$((cat $SYMBOLS \
  | cut -d ' ' -f 1 \
  | sed '/^[a-z]/!d' \
  | sed -r 's/.*/1 2 & &/' \
  ; echo '2') \
  | fstcompile --isymbols=$SYMBOLS --osymbols=$SYMBOLS - \
  | fstclosure -)
        \end{lstlisting}

        The concatenation of the two accepts a capitalized word:
        \begin{lstlisting}[language=bash]
fstconcat <(echo $upper_fst) <(echo $lower_star_fst) > 1c.fst
        \end{lstlisting}

      \item \texttt{word\_fst} accepts a string of zero or more letters in $L$
        excluding <space> and <eps>. It is formed by taking the 1(a) acceptor
        (any single letter in $L$), differencing it with the 1(b) acceptor
        (single <space>), and taking the closure:
        \begin{lstlisting}[language=bash]
word_fst=$(fstdifference 1a.fst 1b.fst \
  | fstclosure)
        \end{lstlisting}

        \texttt{a\_fst} accepts the single character ``a'':
        \begin{lstlisting}[language=bash]
a_fst=$(echo '0 1 a a\n1' \
  | fstcompile --isymbols=$SYMBOLS --osymbols=$SYMBOLS -)
        \end{lstlisting}

        Concatenating $\texttt{word\_fst} \circ \texttt{a\_fst} \circ \texttt{word\_fst}$
        yields an FST accepting a word containing the letter ``a''
        \begin{lstlisting}[language=bash]
fstconcat <(echo $a_fst) <(echo $word_fst) \
  | fstconcat <(echo $word_fst) - \
  > 1d.fst
        \end{lstlisting}

    \end{enumerate}

  \item We first remove $\epsilon$s, determinize, and minimize all FSTs from Question 1:
    \begin{lstlisting}[language=bash]
epsdetmin() {
  fst=$1
  cat $fst.fst \
    | fstrmepsilon \
    | fstdeterminize \
    | fstminimize \
    > $fst.min.fst
}
for fst in '1a' '1b' '1c' '1d'; do
  epsdetmin $fst
done
    \end{lstlisting}
    This makes future FST operations efficient.

    \begin{enumerate}[label=(\alph*)]
      \item
        We concatenate the 1(c) acceptor (capitalized word) with
        the Kleene plus of the 1(b) acceptor (one or more spaces)
        to yield an acceptor for a single capitalized word followed
        by one or more spaces. Taking its Kleene closure yields an
        acceptor for zero or more words followed by spaces:
        \begin{lstlisting}[language=bash]
fstclosure --closure_plus 1b.min.fst \
  | fstconcat 1c.min.fst - \
  | fstclosure - \
  > 2a.fst
        \end{lstlisting}
        There are 5 states with 57 arcs before \texttt{epsdetmin} ($\epsilon$-removal,
        determinize, minimize) and 3 states with 80 arcs after.

      \item Since 1(c) accepts a word beginning in a capital letter, unioning
        1(c) with its reverse yields an acceptor for a word either beginning or
        ending a capitalized letter:
        \begin{lstlisting}[language=bash]
fstreverse 1c.min.fst \
  | fstunion 1c.min.fst - \
  > 2b.fst
        \end{lstlisting}
        There are 5 states with 106 arcs before \texttt{epsdetmin} and 4
        states with 130 arcs after.

      \item Intersecting 1(c) (capitalized word) with 1(d) (letter ``a'') yields
        an acceptor for a word that is capitalized and contains the letter ``a'':
        \begin{lstlisting}[language=bash]
fstintersect 1c.min.fst 1d.min.fst \
  > 2c.fst
        \end{lstlisting}
        There are 3 states with 78 arcs both before and after \texttt{epsdetmin}.

      \item \texttt{word\_fst} accepts zero or more letters in $L$ (excluding
        <space>):
        \begin{lstlisting}[language=bash]
word_fst=$(fstdifference 1a.min.fst 1b.min.fst \
  | fstclosure -)
        \end{lstlisting}

        Taking the complement of the 1(d) acceptor (letter ``a'') with respect to
        \texttt{word\_fst} yields \texttt{no\_a\_fst}, an acceptor for words
        not containing ``a'':
        \begin{lstlisting}[language=bash]
no_a_fst=$(fstdifference <(echo $word_fst) 1d.min.fst)
        \end{lstlisting}

        Unioning \texttt{no\_a\_fst} (words not containing ``a'') with
        the 1(c) acceptor (capitalized words) yields an acceptor for
        a word that is capitalized or does not contain an ``a'':
        \begin{lstlisting}[language=bash]
fstunion 1c.fst <(echo $no_a_fst) \
  > 2d.fst
        \end{lstlisting}
        There are 4 states with 104 arcs before \texttt{epsdetmin} and 3
        states with 77 arcs after.

      \item Another way to implement the 2(d) FST is to use De Morgan's law and
        the law of double negation, which together imply
        \begin{align}
          \texttt{cap} \lor \texttt{no\_a} \equiv
          \lnot ((\lnot \texttt{cap}) \land (\lnot \texttt{no\_a}))
        \end{align}

        We first form \texttt{not\_cap\_fst}, which implements $\lnot
        \texttt{cap}$, by taking the complement of the 1(c) acceptor
        (capitalized word) with respect to \texttt{word\_fst} (zero or
        more letters from $L$ excluding <space>):
        \begin{lstlisting}[language=bash]
not_cap_fst=$(fstdifference <(echo $word_fst) 1c.min.fst)
        \end{lstlisting}

        Intersecting \texttt{not\_cap\_fst} with the FST from 1(d) (letter
        ``a'') yields an implementation of $\lnot ((\lnot \texttt{cap}) \land
        (\lnot \texttt{no\_a}))$, which we can complement with respect to
        \texttt{word\_fst} to yield an acceptor for a word that is capitalized
        or does not contain an ``a'': \begin{lstlisting}[language=bash]
fstintersect <(echo $not_cap_fst) 1d.min.fst \
  | fstdifference <(echo $word_fst) - \
  > 2e.fst
        \end{lstlisting}

        % TODO: rerun
        There are 2 states with 51 arcs both before and after \texttt{epsdetmin}.
        This is less states and arcs than the 2(d) acceptor implementing the
        same regular expression.
    \end{enumerate}

  \item We first implement a transducer mapping nonzero numbers to their English form and
    ignores zeros. \texttt{nonzero} defines a transducer mapping a single numeric
    digit to its English form:
    \begin{lstlisting}[language=bash]
cat > nonzero.txt <<EOF
0 1 1 one
0 1 2 two
0 1 3 three
0 1 4 four
0 1 5 five
0 1 6 six
0 1 7 seven
0 1 8 eight
0 1 9 nine
1
EOF
    \end{lstlisting}
    \texttt{zero} defines a transducer consuming a ``0'' from the input
    without producing anything:
    \begin{lstlisting}[language=bash]
cat > zero.txt <<EOF
0 1 0 <eps>
1
EOF
    \end{lstlisting}
    We can then union \texttt{zero} and \texttt{nonzero} to yield
    the desired FST:
    \begin{lstlisting}[language=bash]
fstunion zero.fst nonzero.fst digits.fst
    \end{lstlisting}

    To handle numbers in $11-19$, \texttt{teens1} consumes a single ``1''
    (the first of two digits) without producing any output:
    \begin{lstlisting}[language=bash]
cat > teens1.txt <<EOF
0 1 1 <eps>
1
EOF
    \end{lstlisting}
    \texttt{teens2} consumes a single digit (the second digit after seeing a ``1'')
    and outputs the correct English read form for the pair:
    \begin{lstlisting}[language=bash]
cat > teens2.txt <<EOF
0 1 0 ten
0 1 1 eleven
0 1 2 twelve
0 1 3 thirteen
0 1 4 fourteen
0 1 5 fifteen
0 1 6 sixteen
0 1 7 seventeen
0 1 8 eighteen
0 1 9 nineteen
1
EOF
    \end{lstlisting}
    Concatenating the two yields \texttt{teens}, an FST mapping digits in $11-19$
    to their English read form:
    \begin{lstlisting}[language=bash]
fstconcat teens1.fst teens2.fst \
  > teens.fst
    \end{lstlisting}

    We now handle the two digit pairs $20-99$
    \texttt{tens} handles the first of two digits:
    \begin{lstlisting}[language=bash]
cat > tens1.txt <<EOF
0 1 2 twenty
0 1 3 thirty
0 1 4 forty
0 1 5 fifty
0 1 6 sixty
0 1 7 seventy
0 1 8 eighty
0 1 9 ninety
1
EOF
    \end{lstlisting}
    Concatenating with \texttt{digits} yields the desired FST (\texttt{tens}):
    \begin{lstlisting}[language=bash]
fstconcat tens1.fst digits.fst \
  > tens.fst
    \end{lstlisting}

    To handle two digit pairs $01-09$, we concatenate \texttt{zero}
    with \texttt{nonzero}. Unioning the result with \texttt{tens}
    and \texttt{teens} yields a transducer for all nonzero pairs $01-99$ (\texttt{pair\_nonzero\_fst}):
    \begin{lstlisting}[language=bash]
pair_nonzero_fst=$(fstunion \
  teens.fst \
  <(fstconcat zero.fst nonzero.fst) \
  | fstunion - tens.fst)
    \end{lstlisting}

    We would like to reuse \texttt{pair\_nonzero\_fst} to transduce the thousands place i.e.\
    the first 2 of the 5 digits. To ensure that we only output ``thousand'' when it is non-zero,
    we construct \texttt{pair\_zero\_fst} to recognize a pair of zeros:
    \begin{lstlisting}[language=bash]
pair_zero_fst=$(fstconcat zero.fst zero.fst)
    \end{lstlisting}
    and \texttt{append\_thousand} to emit ``thousand'' without consuming anything:
    \begin{lstlisting}[language=bash]
cat > append_thousand.txt <<EOF
0 1 <eps> thousand
1
EOF
    \end{lstlisting}
    \texttt{thousand} can now be implemented by concatenating
    \texttt{pair\_nonzero\_fst} with \texttt{append\_thousand}
    to append the ``thousand'' suffix when it is nonzero and
    unioning with \texttt{pair\_zero\_fst} to emit nothing
    when it is zero:
    \begin{lstlisting}[language=bash]
fstconcat <(echo $pair_nonzero_fst) <(cat append_thousand.fst) \
  | fstunion - <(echo $pair\_zero\_fst) \
  > thousand.fst
    \end{lstlisting}

    After handling the thousands place, we next handle the hundreds. \texttt{append\_hundred}
    consumes nothing and outputs ``hundred'':
    \begin{lstlisting}[language=bash]
cat > append_hundred.txt <<EOF
0 1 <eps> hundred
1
EOF
    \end{lstlisting}
    \texttt{hundred} transduces a single digit, appending a ``hundred'' suffix only if
    the digit is nonzero:
    \begin{lstlisting}[language=bash]
fstconcat nonzero.fst append_hundred.fst \
  | fstunion - zero.fst \
  > hundred.fst
    \end{lstlisting}

    After handling the hundreds, the tens and ones place (\texttt{ten}) can
    simply union \texttt{pair\_nonzero\_fst} with \texttt{pair\_zero\_fst}:
    \begin{lstlisting}[language=bash]
(echo $pair_nonzero_fst) \
  > ten.fst
    \end{lstlisting}

    However, the string \texttt{00000} is a corner case not yet handled. To include
    this, we implement \texttt{zero\_explicit} to recognize this case and output
    ``zero'':
    \begin{lstlisting}[language=bash]
cat > zero_explicit.txt <<EOF
0 1 0 zero
1
EOF
fstconcat zero.fst zero.fst \
  | fstconcat - zero.fst \
  | fstconcat - zero.fst \
  | fstconcat - zero_explicit.fst \
  > all_zero.fst
    \end{lstlisting}

    The final model is obtained by first concatenating \texttt{thousand},
    \texttt{hundred}, and \texttt{ten} to yield a transducer for
    digits within $00001-99999$. \texttt{zero\_explicit} is then
    unioned in to handle the $00000$ case:
    \begin{lstlisting}[language=bash]
model_fst=$(fstconcat thousand.fst hundred.fst \
  | fstconcat - ten.fst \
  | fstunion - all_zero.fst)
    \end{lstlisting}

    We test our model on some example inputs and verify the results:
    \begin{lstlisting}[language=bash]
test_io() {
  input_fst=$(printf "%s\n" \
    "0 1 $1 $1"\
    "1 2 $2 $2"\
    "2 3 $3 $3"\
    "3 4 $4 $4"\
    "4 5 $5 $5"\
    "5" \
    | fstcompile --isymbols=$SYMBOLS --osymbols=$SYMBOLS -)

  result_fst=$(fstcompose <(echo $input_fst) <(echo $model_fst) \
    | fstproject --project_output)

  print "Input:"
  fstprint --isymbols=$SYMBOLS --osymbols=$SYMBOLS <(echo $input_fst)
  print "Result:"
  fstprint --isymbols=$SYMBOLS --osymbols=$SYMBOLS <(echo $result_fst)
}
    \end{lstlisting}

    \begin{lstlisting}[language=bash]
test_io 0 0 0 0 0
Result:
0       1       <eps>   <eps>
0       2       <eps>   <eps>
1       3       <eps>   <eps>
2       4       <eps>   <eps>
3       5       <eps>   <eps>
4       6       <eps>   <eps>
5       7       <eps>   <eps>
6       8       <eps>   <eps>
7       9       <eps>   <eps>
8       10      <eps>   <eps>
9       11      <eps>   <eps>
10      12      <eps>   <eps>
11      13      <eps>   <eps>
12      14      <eps>   <eps>
13      15      <eps>   <eps>
14      16      <eps>   <eps>
15      17      <eps>   <eps>
16      18      <eps>   <eps>
17      19      <eps>   <eps>
18      20      zero    zero
19      21      <eps>   <eps>
20
21      22      <eps>   <eps>
22
    \end{lstlisting}

    \begin{lstlisting}[language=bash]
test_io 0 0 0 2 4
Result:
0       1       <eps>   <eps>
1       2       <eps>   <eps>
2       3       <eps>   <eps>
3       4       <eps>   <eps>
4       5       <eps>   <eps>
5       6       <eps>   <eps>
6       7       <eps>   <eps>
7       8       <eps>   <eps>
8       9       <eps>   <eps>
9       10      twenty  twenty
10      11      <eps>   <eps>
11      12      <eps>   <eps>
12      13      four    four
13
    \end{lstlisting}
    \begin{lstlisting}[language=bash]
test_io 0 0 1 0 0
Result:
0       1       <eps>   <eps>
1       2       <eps>   <eps>
2       3       <eps>   <eps>
3       4       <eps>   <eps>
4       5       <eps>   <eps>
5       6       one     one
6       7       <eps>   <eps>
7       8       hundred hundred
8       9       <eps>   <eps>
9       10      <eps>   <eps>
10      11      <eps>   <eps>
11      12      <eps>   <eps>
12      13      <eps>   <eps>
13
    \end{lstlisting}
    \begin{lstlisting}[language=bash]
test_io 0 1 1 0 6
Result:
0       1       <eps>   <eps>
1       2       <eps>   <eps>
2       3       <eps>   <eps>
3       4       one     one
4       5       <eps>   <eps>
5       6       thousand        thousand
6       7       <eps>   <eps>
7       8       one     one
8       9       <eps>   <eps>
9       10      hundred hundred
10      11      <eps>   <eps>
11      12      <eps>   <eps>
12      13      <eps>   <eps>
13      14      <eps>   <eps>
14      15      six     six
15
    \end{lstlisting}
    \begin{lstlisting}[language=bash]
test_io 4 0 0 0 1
Result:
0       1       <eps>   <eps>
1       2       forty   forty
2       3       <eps>   <eps>
3       4       <eps>   <eps>
4       5       <eps>   <eps>
5       6       thousand        thousand
6       7       <eps>   <eps>
7       8       <eps>   <eps>
8       9       <eps>   <eps>
9       10      <eps>   <eps>
10      11      <eps>   <eps>
11      12      <eps>   <eps>
12      13      <eps>   <eps>
13      14      one     one
14
    \end{lstlisting}

  \item
    \begin{enumerate}[label=(\alph*)]
      \item We create a flower transducer \texttt{rot13} implementing the $rot13$ cipher:
        \begin{lstlisting}[language=bash]
cat > rot13.txt <<EOF
0 0 <space> <space>
0 0 . .
0 0 , ,
0 0 a n
0 0 b o
...(truncated)
0 0 z m
0
EOF
        \end{lstlisting}

      \item We use a loop to create an acceptor \texttt{message} for the message:
        \begin{lstlisting}[language=bash]
message="my secret message"
for ((i=1; i <= ${#message}; i++)); do
  if [[ $message[i] == ' ' ]]; then
    echo $((i-1)) $i "<space>" "<space>" >> message.txt
  else
    echo $((i-1)) $i $message[i] $message[i] >> message.txt
  fi
done
echo $((i-1)) >> message.txt
    \end{lstlisting}

    Composing with the \texttt{rot13} transducer gives the encoded message:
    \begin{lstlisting}[language=bash]
fstcompose message.fst rot13.fst \
  > encode.fst
fstprint --isymbols=$SYMBOLS --osymbols=$SYMBOLS encode.fst

    Encode:
    0       1       m       z
    1       2       y       l
    3       3       <space> <space>
    3       4       s       f
    4       5       e       r
    5       6       c       p
    6       7       r       e
    7       8       e       r
    8       9       t       g
    9       10      <space> <space>
    10      11      m       z
    11      12      e       r
    12      13      s       f
    13      14      s       f
    14      15      a       n
    15      16      g       t
    16      17      e       r
    17
    \end{lstlisting}

    Since \texttt{rot13} is an involution, we can compose the encoded
    message again with \texttt{rot13} to decode:
    \begin{lstlisting}[language=bash]
fstcompose encode.fst rot13.fst \
  > decode.fst
fstprint --isymbols=$SYMBOLS --osymbols=$SYMBOLS decode.fst

    Decode:
    0       1       m       m
    1       2       y       y
    2       3       <space> <space>
    3       4       s       s
    4       5       e       e
    5       6       c       c
    6       7       r       r
    7       8       e       e
    8       9       t       t
    9       10      <space> <space>
    10      11      m       m
    11      12      e       e
    12      13      s       s
    13      14      s       s
    14      15      a       a
    15      16      g       g
    16      17      e       e
    17
    \end{lstlisting}

  \item We first define the \texttt{rot16} transducer:
    \begin{lstlisting}[language=bash]
cat > rot16.txt <<EOF
0 0 <space> <space>
0 0 . .
0 0 , ,
0 0 a q
0 0 b r
...(truncated)
0 0 z p
0
EOF
    \end{lstlisting}

    To permit both transductions, we take the union and the closure
    to form \texttt{rot1316\_decoder}:
    \begin{lstlisting}[language=bash]
rot1316_decoder=$(fstunion rot13.fst rot16.fst \
  | fstclosure \
  | fstrmepsilon)
    \end{lstlisting}

    \begin{lstlisting}[language=bash]
encoded=$DIR/4.encoded1.fst
fstcompose $encoded <(echo $rot1316_decoder) \
  | fstproject --project_output \
  > result.fst
    \end{lstlisting}
    There are 731 states with 1458 arcs before \texttt{epsdetmin} and
    366 states with 649 arcs after.

    From examining the encoded message \texttt{encoded}, we see that there are
    284 possible characters.  Since each character can either be encoded using
    $rot13$ or $rot16$, there are a total of $2^284$ possible distinct strings
    through the FST!


  \item Composing with the language model:
    \begin{lstlisting}[language=bash]
LM=$DIR/4.lm.fst
fstcompose result2.fst $LM \
  > result_lm.fst
    \end{lstlisting}
    yields the decoding:
    \begin{quote}
      whether i shall turn out to be the hero of my own life, or whether that
      station will be held by anybody else, these pages must show. to begin my
      life with the beginning of my life, i record that i was born as i have
      been informed and believe on a friday, at twelve oclock at night. it was
      remarked that the clock began to strike, and i began to cry,
      simultaneously.
    \end{quote}

  \item We construct a FST which swaps two characters.  For every
    possible pair of letters, the FST is taken from $0$ to a unique state
    between $100-125$ after reading the two letters. The only path from this
    state to the end state ($99$) outputs the two letters reversed. Since only
    some pairs of consecutive letters are swapped, thre is also a path directly
    from the start state to the end state to handle unswapped pairs. For single
    letter inputs, the transducer simply acts as the identity.
    \begin{lstlisting}[language=python]
#!/usr/bin/python

def make_fst(fstName):
    with open(fstName + '.txt', 'w') as f:
        f.write("""0 99 <space> <space>
0 99 . .
0 99 , ,
""")
        for i,c in enumerate([chr(k) for k in range(ord('a'), ord('z')+1)]):
            f.write("0 {0} {1} {1}\n".format(99, c)) # pass-through
            f.write("0 {0} {1} <eps>\n".format(i+1, c)) # swap
            for d in [chr(k) for k in range(ord('a'), ord('z')+1)]:
                f.write("{0} {1} {2} {3}\n".format(i+1, i+100, d, d))
                f.write(" {0} 99 <eps> {1}\n".format(i+100, c))
        f.write("99\n")

if __name__ == "__main__":
    import argparse

    parser = argparse.ArgumentParser(description='Generate FST description for swapping letters')
    parser.add_argument('fstName', metavar='N', type=str,
                               help='name of the .txt file to output')
    args = parser.parse_args()
    make_fst(args.fstName)
    \end{lstlisting}

    Using this Python script (\texttt{swap\_letter\_fst.py}), we define a transducer
    for swapping pairs of letters (\texttt{letter\_swap}). We then form
    $\texttt{encoded} \circ \texttt{rot13} \circ \texttt{letter\_swap}^*$ which
    first performs $rot13$ followed by swappnig some pairs of consecutive letters.
    \begin{lstlisting}[language=bash]
encoded=$DIR/4.encoded2.fst

python swap_letter_fst.py 'letter_swap'

(cat $encoded) \
  | fstcompose -  rot13.fst \
  | fstcompose - <(fstclosure letter_swap.fst) \
  | fstproject --project_output \
  | fstrmepsilon result.fst | fstdeterminize | fstminimize - \
  | fstcompose - $LM \
  > result2_lm.fst
    \end{lstlisting}
    Decoding the text yields:
    \begin{quote} %TODO: are the two really the same decoding?
      whether i shall turn out to be the hero of my own life, or whether that station
      will be held by anybody else, these pages must show. to begin my life
      with the beginning of my life, i record that i was born as i have been
      informed and believe on a friday, at twelve oclock at night. it was
      remarked that the clock began to strike, and i began to cry,
      simultaneously.
    \end{quote}
  \end{enumerate}
\end{enumerate}

%\bibliographystyle{alpha}
%\nocite{*}
%\bibliography{refs}

%\appendix

%\section{Code Listings}

\end{document}
